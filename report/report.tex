\documentclass[12pt]{article}
\usepackage[utf8]{inputenc}
\usepackage[
top=2cm,
bottom=2cm,
left=3cm,
right=2cm,
headheight=17pt, % as per the warning by fancyhdr
includehead,includefoot,
heightrounded, % to avoid spurious underfull messages
]{geometry} 
\geometry{a4paper}
\usepackage[english]{babel}
\usepackage{listings}
\usepackage{fancyhdr}
\usepackage{siunitx}
\usepackage{float}
\usepackage{graphicx}
\usepackage{caption}
\usepackage[table]{xcolor}
\usepackage{diagbox}
\usepackage{lipsum}
\usepackage{lineno}

% Assembler
\lstdefinelanguage
{Assembler} % based on the "x86masm" dialect
% with these extra keywords:
{morekeywords={call, mov}} % etc.

% Lecture Name, exercise number, group number/members
\newcommand{\lecture}{Robotics Practical}
\newcommand{\nao}{NAO} % capital letters as official usage
\newcommand{\project}{Walking \nao}
%\newcommand{\groupnumber}{}
\newcommand{\groupmembersshort}{Barth, Wagner}
\newcommand{\groupmemberslist}{Barth, Alexander\\Wagner, Royden}
\newcommand{\semester}{Summer term 2020}
\newcommand{\supervisor}{Wittlinger, Peter}

\fancyhf{}
%\fancyhead[L]{\groupnumber}
\fancyhead[C]{\textsc{\groupmembersshort}}
\fancyfoot[L]{\lecture: \project}
\fancyfoot[R] {\thepage}
\renewcommand{\headrulewidth}{0.4pt}
\renewcommand{\footrulewidth}{0.4pt}
\pagestyle{fancy}

\begin{document}
	\begin{titlepage}
		\centering
		
		{\scshape\LARGE Universität Heidelberg\\Institute for Computer Engineering (ZITI) \par}
		\vspace{1.5cm}
		{\scshape\Large Master of Science Computer Engineering \par}
		\vspace{0.5cm}
		{\scshape\Large \lecture \par}
		\vspace{1.5cm}
		{\huge\bfseries \project \par}
		\vspace{2cm}
		{\large \groupmemberslist \par}
		\vspace{2cm}
		{\itshape supervised by \supervisor \par}
		\vfill
		
		
		% Bottom of the page
		{\large \semester \par}
	\end{titlepage}

\noindent \textbf{Abstract}

@todo: complete

This report describes the walking \nao{} project of the \lecture{} lecture.
This project is done by \groupmembersshort, which are all students in the master program computer engineering.
The main target is to

\tableofcontents
\newpage

\section{Testing and prep}

@todo: ...\\

test development env for ROS (in docker?)\\
try nao/ naoqi with ROS: http://wiki.ros.org/nao\\
use DenseDepth as basemodel for depth estimation: https://arxiv.org/pdf/1812.11941v2.pdf\\
write test-funcs to showcase depth estimation model in realtime via your webcam

\section{Remote control}

@todo: ...\\

construct NAO-nano-backpack\\
write webapp to control NAO\\
serve webapp with flask on nano\\

idea: use UI5

\section{Depth estimation}

@todo: ...\\

use opencv to detect if dark blue colored objects are in the desired direction and decline movement if so\\

get desired direction and draw on depth image\\
use color picker to check if movement in chosen direction is possible (dark blue == not possible)\\
try to implement a color search algo in opencv similar to SSD (single shot detector)\\

implement functions to check if navigation in certain direction is possible based on point cloud of surroundings\\
optional: try to change basemodel of DensDepth to MobileNet V1/V2 to increase speed

\section{Commissioning}

@todo: instructions for commissioning

\section{Conclusion and outlook}

@todo: future usage in lecturing and the robotic lab

\end{document}