%---------------------------Dokumentkenndaten ANFANG-------------------------%

\documentclass[12pt,a4paper,oneside,ngerman,draft=off,headings=small, fleqn]{scrreprt}
\author{Bart, Prigann, Wagner}
\title{Robotics practical report}
\date{\today}

%---------------------------Dokumentkenndaten ENDE-------------------------%

%---------------------------Paketnutzungen ANFANG-------------------------%

\usepackage[english, ngerman]{babel}
\usepackage[table]{xcolor}%Farbfüllung Tabelle
\usepackage{graphicx}
\usepackage{csquotes}
\usepackage{pdfpages}
\usepackage{setspace}
\usepackage{siunitx}
\sisetup{per-mode=fraction,decimalsymbol=comma}
\usepackage{scrhack}
\usepackage{pifont}
\usepackage{scrlayer-scrpage}
\usepackage[left=25mm, right=25mm, top=25mm, bottom=25mm, head=22pt]{geometry}%Seiteneinrichtung, 25mm Rand allseits, spezifisch für Seite mit renewcommand
\usepackage{fontspec}
\usepackage{lipsum}
\usepackage{float} %Für Gleitobjekte, damit direkt in Text
\usepackage[labelfont={bf}]{caption} %Caption-Schriftstil Deklaration
\usepackage{listings}
\usepackage{xcolor}
\usepackage{multicol}
\usepackage{multirow}
\usepackage{tikz}
\usepackage{amsmath}
\usepackage{amssymb}
\usepackage{subfig}
\usetikzlibrary{calc,arrows.meta,positioning}
\usepackage[all]{nowidow}
\clubpenalty = 10000
\widowpenalty = 10000
\displaywidowpenalty = 10000
\tikzset{
	rect/.style={draw, rectangle, minimum width=2.0cm},
	circ/.style={draw, circle},
	every label/.style={draw=none,font=\scriptsize},
}

\def\circledarrow#1#2#3{ % #1 Style, #2 Center, #3 Radius
	\draw[#1] (#2) +(80:#3) arc(80:-260:#3);
}
\lstdefinestyle{mystyle}{ 
	commentstyle=\color{mGreen},
	keywordstyle=\color{darkpowderblue},
	keywordstyle=[2]\color{darkpink},
	keywordstyle=[3]\color{darkscarlet},
	keywordstyle=[4]\color{indigo(dye)},
	numberstyle=\tiny\color{lightgray},
	stringstyle=\color{red},
	basicstyle=\footnotesize,
	breakatwhitespace=false,         
	breaklines=true,                 
	%captionpos=b,                    
	keepspaces=true,                 
	numbers=left,                    
	numbersep=5pt,
	numberstyle=\tiny,                  
	showspaces=false,                
	showstringspaces=false,
	showtabs=false,                  
	tabsize=2,
	xleftmargin= 25pt,
}
\lstset{escapechar=@,style=mystyle}
\PassOptionsToPackage{hyphens}{url}\usepackage[pdfborderstyle={/S/U/W 1}]{hyperref}
\usepackage[noabbrev]{cleveref}
\setlength{\emergencystretch}{45pt}
\newcommand{\cmark}{\ding{51}}% Haken
\newcommand{\xmark}{\ding{53}}%	x

%---------------------------Paketnutzungen ENDE-------------------------%

%---------------------------LAYOUT Fußzeilen ANFANG-------------------------%

\pagestyle{scrheadings}
\renewcommand{\footfont}{\setmainfont{Arial}\normalfont} %Text in Fußzeile in Arial und 12pt
\setlength{\footheight}{21.75pt}
\setkomafont{pagenumber}{\normalfont} %Seitenzahl in Kopfzeile in Arial und 12pt

%---------------------------LAYOUT Fußzeilen ENDE-------------------------%

%---------------------------LAYOUT Überschriften ANFANG-------------------------%

\setcounter{secnumdepth}{3} %Überschriften bis subsection werden nummeriert

%Ab hier: Definition, dass alle Abschnittsüberschriften in Arial und den jeweiligen Abschnittseinstellungen formatiert sind

\setkomafont{disposition}{\normalfont \bfseries}

%---------------------------LAYOUT Überschriften ENDE-------------------------%

%---------------------------LAYOUT Fließtext ANFANG-------------------------%

\setmainfont{Arial} %Arial als Hauptschriftart (mainfont)
\renewcommand{\baselinestretch}{1.5} %1.5pt Zeilenabstand
\setlength{\parindent}{0pt} %kein Einzug nach Absatz
\setlength{\parskip}{10pt} %Zeilenabstand nach Absatz 10pt
\setkomafont{caption}{\fontsize{9pt}{\baselineskip} \bfseries} %Bild-/Tabellenunterschriften: fett, 9pt

%---------------------------LAYOUT Fließtext ENDE-------------------------%


\begin{document}
	
	%---------------------------TITELBLATT ANFANG-------------------------%
	\iffalse
	\begin{titlepage}
		\includepdf{include/title}
	\end{titlepage}
	\fi
	%---------------------------TITELBLATT ENDE-------------------------%
	
	%---------------------------EINSTELLUNG KopfFußzeile ANFANG-------------------------%
	
	\setmainfont{Arial} %Arial als Schriftart
	\clearscrheadfoot
	\setkomafont{pageheadfoot}{\normalfont}
	\setlength{\headheight}{10mm}
	\ohead{\pagemark\normalfont}
	\ihead{Universität Heidelberg}
	
	\cfoot{MScTI\_ROBP}
	\ofoot{SS 2020}
	\ifoot{Barth, Prigann, Wagner}
	
	%---------------------------EINSTELLUNG KopfFußzeile ENDE-------------------------%
	
	%---------------------------HAUPTTEIL ANFANG-------------------------%
		
	\chapter{Ebene}
	\thispagestyle{scrheadings}
	
	\lipsum[1]
	
	\section{Ebene}
	
	\lipsum[1]
	
	\subsection{Ebene}
	
	\begin{figure}[H]
		\centering
		%\includegraphics[width=0.9\linewidth]{include/test}
		\caption{Beschriftung}
		\label{fig:test}
	\end{figure}

	%---------------------------HAUPTTEIL ENDE-------------------------%
	
	%---------------------------ZWEITER HAUPTTEIL ANFANG-------------------------%
	
	\chapter{Main}
	\thispagestyle{scrheadings}
	
	\cref{tab:test} bla.
	
	\begin{minipage}{\linewidth}
		\begin{table}[H]
			\centering
			\captionabove{Beschriftung}
			\begin{tabular}{c|c}
				\hline
				\cellcolor{gray!40}\textbf{Dies} &\cellcolor{gray!40}\textbf{Das}\\
				\hline\hline
				Item & \xmark\\\hline
				Item & \cmark\\\hline
			\end{tabular}
			\label{tab:test}
		\end{table}
	\end{minipage}

	\lipsum[1]
	
	%---------------------------ZWEITER HAUPTTEIL ENDE-------------------------%
	
	%---------------------------Fazit & Ausblick ANFANG-------------------------%
	
	\chapter{Ausblick}\label{cha:ausb}
	\thispagestyle{scrheadings}
	
	Ausblick für die Verwendung des Projekts in zukünftigen Projekten/Lehre
	
	%---------------------------Fazit & Ausblick ENDE-------------------------%

\end{document}